I am writing in response to the advertised faculty position (\textbf{\appPosition{} \appJobID}) in the \appDept{} at the \appSchool{}. I am currently a \myTitle{} at the \myDept{} of the \mySchoolShort{}, supervised by \mySuper{}. I obtained my \myDegree{} degree in \myMajor{} from the \myPhDSchool{}, in 2021, co-advised by \myAdvisor{} and \myCoAdvisor{}. I have established a diverse yet synergetic research portfolio centered around \textbf{silicon photonics (SiPh) optical interconnects}, \appSpecific{}. I believe that my interdisciplinary research background will make me a great fit to contribute to the continued excellence of your department.

My research is strongly motivated by the prevalence of communication bottleneck in the era of data ubiquity\textemdash emerged in modern distributed computing infrastructures and exacerbated by the pervasiveness of data-intensive AI/machine learning applications\textemdash that calls for a transformative change in the underlying system connectivity to allow for future scalability. Aiming to fully unleash the potential of CMOS-compatible silicon photonics for ultra high-bandwidth and energy-efficient optical interconnects, I have worked toward building a design ecosystem around a unique link architecture that can utilize an unprecedentedly broad optical band, allowing for massively scalable channel parallelism.

Serving as the foundation of the ecosystem, my Ph.D. research was dedicated to \textbf{vertically integrating design enablement technologies} for integrated silicon photonics across device, circuit, and system levels, including the accurate and efficient modeling and simulation methodologies developed for various abstract levels, as well as process variation characterization and mitigation techniques for yield and performance optimization. These efforts led to publications in top design automation conferences, such as DAC and ICCAD, as well as renowned photonics publication venues, such as OFC and JLT, reaching audiences from both electronics and photonics communities.

Building on top of and orthogonal to my Ph.D. research, I have worked, as a postdoc at Columbia University, toward the design implementation and system application of massively scalable silicon photonics chip I/O, which involves \textbf{horizontally integrating optimization techniques} across the entire design cycle from device and circuit design to post-fabrication tuning and system runtime reconfiguration. I have led the development of two generations of a SiPh transceiver chip\textemdash co-designed with 3D-integrated electronic drivers and leading packaging solutions through deep collaboration with both academia and industry partners\textemdash featuring an 8\,mm \texttimes{} 8\,mm footprint tightly integrated with over 2,000 microresonator devices on a single chip that targets 16\,Tbps chip I/O bandwidth at sub-pJ/b energy consumption. The featured link architecture served as one of the enabling technologies in an awarded SRC JUMP 2.0 grant (23 PIs, \$35M) for which I contributed to the proposal writing, and a manuscript is being prepared for an invited submission to \emph{Nature Communications Physics} describing the chip design and characterization. System-level reconfigurable architectures and runtime reconfiguration techniques were also explored, enabled by and leveraging the unique link architecture, to further improve the performance and energy efficiency of optically connected computing systems by equipping them with traffic adaptability.

I envision new opportunities that can emerge from the increasingly tight integration of silicon photonics with the computing electronics, which relies on the synergetic interplay between innovative link/system architectures and enhanced design capabilities. For example, a paradigm change can be pursued for future chip-to-chip connectivity by leveraging 3D optical I/Os densely routed with multiple layers of waveguides on-chip\textemdash a vision that I helped formulate for presenting at the 3D Optoelectronics Workshop at the 2023 DARPA ERI Summit\textemdash opening up new horizons for
computing architectures and interconnect functionalities with manifolded reach of on-board optical connectivity, while accompanied by new yet exciting challenges to be tackled regarding design and optimization. I also recognize the rapidly evolving data landscape in distributed computing infrastructures driven by data privacy concerns\textemdash which instigated the emergence of privacy-centric computing frameworks, such as federated learning which exchange models instead of data\textemdash as well as the surging trend in the adoption of large deep-learning models, such as GPTs, which jointly tilt the balance of computation-communication further to the latter, creating greater potential for traffic-adaptable optical interconnects to play a more significant role in future computing systems. Due to the interdisciplinary nature of my research aiming at the intersections of electronics/photonics, devices/systems, and design/applications, I look forward to collaborating with the diverse faculty members at the \appSchool{}%