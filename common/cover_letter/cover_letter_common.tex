I am writing in response to the advertised faculty position (\textbf{\appPosition{} \appJobID}) in the \appDept{} at the \appSchool{}. I am currently a \myTitle{} at the \myDept{} of the \mySchoolShort{}, supervised by \mySuper{}. I obtained my \myDegree{} degree in \myMajor{} from the \myPhDSchool{}, in 2021, co-advised by \myAdvisor{} and \myCoAdvisor{}. I have established a diverse yet synergetic research portfolio centered around silicon photonics (SiPh) optical interconnects, and believe that my interdisciplinary research background will make me a great fit to contribute to the continued excellence of your department.

My research is strongly motivated by the prevalence of communication bottleneck in the era of data ubiquity\textemdash emerged in modern distributed computing infrastructures and exacerbated by the pervasiveness of data-intensive AI/machine learning applications\textemdash that calls for a transformative change in the underlying system connectivity to allow for future scalability. Aiming to fully unleash the potential of CMOS-compatible silicon photonics for ultra high-bandwidth and energy-efficient optical interconnects, I have worked toward building a design ecosystem centered around a unique link architecture that can utilize an unprecedentedly broad optical bandwidth, allowing for massively scalable channel parallelism. Specifically, my Ph.D. research was dedicated to \textbf{vertically integrating design enablement technologies for integrated silicon photonics across device, circuit, and system levels}, including the accurate and efficient modeling and simulation methodologies developed for various abstract levels, as well as process variation characterization and mitigation techniques for yield and performance optimization. The research effort led to publications in top design automation conferences, such as DAC and ICCAD, as well as renowned photonics publication venues, such as OFC and JLT, reaching audiences from both electronics and photonics communities.

Building on top of and orthogonal to my Ph.D. research, I have worked, as a postdoc at Columbia University, toward the design implementation and system application of a massively scalable silicon photonics I/O chip, which involves \textbf{horizontally integrating optimization techniques across the entire design cycle}, encompassing the early design stage, post-fabrication tuning, as well as system runtime reconfiguration. I have led the development of two generations of the SiPh I/O chip\textemdash co-designed with 3D-integrated electronic drivers and leading package solutions through deep collaboration with both academia and industry partners\textemdash featuring an 8\,mm \texttimes{} 8\,mm footprint tightly integrated with over 2000 microresonator devices and targeting 16\,Tbps chip I/O bandwidth at sub-pJ/b energy consumption. The featured link architecture served as one of the fundamental technologies in an awarded SRC JUMP 2.0 grant (23 PIs, \$35M) for which I contributed to the proposal writing, and a manuscript is being prepared for an invited submission to \emph{Nature Communications Physics} describing the chip design and characterization.

I envision new opportunities that can emerge from the increasingly tight integration of photonics with the computing electronics, which relies on the ecosystem that embraces both innovations in link/system architectures and advancements in design capabilities.




