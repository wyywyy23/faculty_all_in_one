I am writing in response to the advertised faculty position (\textbf{\appPosition{} \appJobID}) in the \appDept{} at the \appSchool{}. I am currently a \myTitle{} at the \myDept{} of the \mySchoolShort{} under the mentorship of \mySuper{}. My \myDegree{} in \myMajor{} was awarded by the \myPhDSchool{} in 2021, where I was co-advised by \myAdvisor{} and \myCoAdvisor{}. My research portfolio, which is both diverse and synergistic, focuses on silicon photonics (SiPh) optical interconnects, \appSpecific{}. I believe that my interdisciplinary research background positions me well to contribute to your department's ongoing excellence.

The prevalence of communication bottleneck in the era of data ubiquity\textemdash emerged in modern distributed computing infrastructures and exacerbated by the pervasiveness of data-intensive AI/machine learning applications\textemdash drives my research that seeks \textbf{transformative connectivity solutions} for future system scalability. I aim to cultivate the potential of CMOS-compatible silicon photonics to develop ultra high-bandwidth and energy-efficient optical interconnects. Building upon a unique link architecture that facilitates unprecedented channel parallelism, I envision system-level breakthroughs that boost the density, adaptability, and functionalities of optical interconnects amidst the ever-evolving data landscape.

My Ph.D. research laid the groundwork for this vision, \textbf{vertically integrating design enablement technologies} for integrated silicon photonics across device, circuit, and system levels, including modeling, simulation, and variation mitigation techniques to democratize the design of yield- and performance-optimized SiPh devices and circuits. The research resulted in publications at leading design automation conferences like DAC and ICCAD, esteemed photonics venues like OFC and JLT, and a book chapter in press with Springer, bridging electronics and photonics research communities.

Orthogonally expanding on my Ph.D. work, my postdoctoral work at the Columbia University focuses on the design and implementation of massively scalable SiPh chip I/Os, \textbf{horizontally integrating optimization techniques} across the entire design cycle, from initial design to post-fabrication tuning and runtime reconfiguration. I spearheaded two generations of SiPh transceiver chips\textemdash co-designed with 3D-integrated electronic drivers and advanced packaging in collaboration with both academia and industry partners\textemdash integrating over 2,000 microresonators into an 8\,mm \texttimes{} 8\,mm footprint to achieve 16\,Tbps chip I/O bandwidth with sub-pJ/b energy consumption. The highlighted link architecture played a pivotal role in securing a \$35M SRC JUMP 2.0 grant involving 23 principal investigators, to which I contributed through proposal writing. An invited paper detailing the chip's design and characterization is currently under review for CICC 2024, and a manuscript focusing on the end-to-end demonstration of the link is slated for an invited submission to \emph{Nature Communications Physics}.

Moving forward, I anticipate an even more diverse and dynamic data landscape, shaped by the growing heterogeneity of computing applications. Emerging privacy-focused frameworks such as federated learning emphasize the exchange of models over data, which\textemdash coinciding with the expansion of large models like GPTs\textemdash further pushes the system bottleneck from computation to communication capabilities. This shift necessitates the next generation of \textbf{reconfigurable system connectivity} that can adapt on-demand to varying traffic needs while maintaining high bandwidth and energy efficiency. My foundation work in reconfigurable system architectures and runtime adaptation techniques, built upon the scalable link architecture, will be the cornerstone of my ongoing research endeavor to equip future computing infrastructures for the ever-evolving data context.

I also envision opportunities arising from deeper integration of silicon photonics within computing sockets, dependent on the synergy between \textbf{innovative system architectures} and groundbreaking optical I/O technologies. For instance, by leveraging 3D optical I/Os densely routed



I foresee opportunities arising from the close integration of silicon photonics with computing electronics, dependent on the synergy between innovative architectures and improved design capabilities. For instance, we can revolutionize future chip-to-chip connectivity by leveraging 3D optical I/Os densely routed with multiple layers of waveguides on-chip\textemdash a concept that I helped formulate for presentation at the 2023 DARPA ERI Summit\textemdash opening up new horizons for
computing architectures and interconnect functionalities with manifolded reach of on-board optical connectivity, while signaling new challenges in design and optimization. I also recognize the rapidly evolving data landscape in distributed computing\textemdash influenced by privacy concerns and the rise of frameworks like federated learning, which prioritize model over data exchange, as well as the surge in large models like GPTs\textemdash shifting the computation-communication balance further toward the latter, creating greater potential for traffic-adaptable optical interconnects in future computing systems. My interdisciplinary research, sitting at the nexus of electronics/photonics, devices/systems, and design/applications, positions me to collaborate with the diverse faculty members at the \appSchool{}%