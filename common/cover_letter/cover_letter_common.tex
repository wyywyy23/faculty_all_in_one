I am writing in response to the advertised faculty position (\textbf{\appPosition{} \appJobID}) in the \appDept{} at the \appSchool{}. I am currently a \myTitle{} at the \myDept{} of the \mySchoolShort{} under the mentorship of \mySuper{}. My \myDegree{} in \myMajor{} was awarded by the \myPhDSchool{} in 2021, where I was co-advised by \myAdvisor{} and \myCoAdvisor{}. My research portfolio, which is both diverse and synergistic, focuses on silicon photonics (SiPh) optical interconnects, \appSpecific{}. I believe that my interdisciplinary research background positions me well to contribute to your department's ongoing excellence.

The prevalence of communication bottleneck in the era of data ubiquity\textemdash emerged in modern distributed computing infrastructures and exacerbated by the pervasiveness of data-intensive AI/machine learning applications\textemdash drives my research that seeks \textbf{transformative connectivity solutions} for future system scalability. My goal is to cultivate the potential of CMOS-compatible silicon photonics for ultra high-bandwidth, energy-efficient optical interconnects, focusing on a design ecosystem built around a unique link architecture that supports an unprecedentedly broad optical band, allowing for massively scalable channel parallelism.

My Ph.D. research laid the ecosystem's groundwork, \textbf{vertically integrating design enablement technologies} for integrated silicon photonics across device, circuit, and system levels, including modeling, simulation, and variation mitigation techniques to democratize the design of yield- and performance-optimized SiPh devices and circuits. The research resulted in publications at leading design automation conferences like DAC and ICCAD, esteemed photonics venues like OFC and JLT, and a book chapter in press with Springer, bridging electronics and photonics research communities.

Orthogonally expanding on my Ph.D. work, my postdoc at Columbia focuses on the design and implementation of massively scalable SiPh chip I/O, \textbf{horizontally integrating optimization techniques} across the entire design cycle, from initial design to post-fabrication tuning and runtime reconfiguration. I spearheaded two generations of SiPh transceiver chips\textemdash co-designed with 3D-integrated electronic drivers and advanced packaging in collaboration with both academia and industry partners\textemdash integrating over 2,000 microresonators into an 8\,mm \texttimes{} 8\,mm footprint to achieve 16\,Tbps chip I/O bandwidth with sub-pJ/b energy consumption. The highlighted link architecture contributed to securing an SRC JUMP 2.0 grant with 23 PIs and \$35M funding, to which I added through proposal writing. A manuscript detailing this work is slated for an invited submission to \emph{Nature Communications Physics}. I also explored system-level reconfigurable architectures and runtime techniques, harnessing the unique link architecture to further enhance performance and energy efficiency in optically connected computing systems through traffic adaptability.

I foresee opportunities arising from the close integration of silicon photonics with computing electronics, dependent on the synergy between innovative architectures and improved design capabilities. For instance, we can revolutionize future chip-to-chip connectivity by leveraging 3D optical I/Os densely routed with multiple layers of waveguides on-chip\textemdash a concept that I helped formulate for presentation at the 2023 DARPA ERI Summit\textemdash opening up new horizons for
computing architectures and interconnect functionalities with manifolded reach of on-board optical connectivity, while signaling new challenges in design and optimization. I also recognize the rapidly evolving data landscape in distributed computing\textemdash influenced by privacy concerns and the rise of frameworks like federated learning, which prioritize model over data exchange, as well as the surge in large models like GPTs\textemdash shifting the computation-communication balance further toward the latter, creating greater potential for traffic-adaptable optical interconnects in future computing systems. My interdisciplinary research, sitting at the nexus of electronics/photonics, devices/systems, and design/applications, positions me to collaborate with the diverse faculty members at the \appSchool{}%