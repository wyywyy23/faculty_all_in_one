\section{Mentoring}
My mentoring philosophy extends beyond knowledge sharing and intellectual guidance. I am dedicated to providing holistic support for the academic and professional development of the students. I will strive to create a supportive and collaborative research environment, inspiring students to excel in their studies, uphold professionalism, and explore their own research interests until they emerge as independent researchers. My time as a postdoctoral researcher at the Columbia University allowed me to partially implement this philosophy, guiding several Ph.D. students to significant milestones, including publishing their first papers as lead authors at premier conferences like OFC and CLEO, and securing their initial industry positions. I eagerly anticipate the opportunity to extend and refine my mentorship practices at the \appSchool{}.

\section{Post\textendash COVID-19 Considerations}

In the post\textendash COVID-19 era, I have adapted to the emerging norms of hybrid teaching and research settings. I am prepared to further refine my teaching and mentoring approaches to accommodate these evolving challenges. For example, I plan to enhance my slides with additional annotations and essential animations to offset the lack of real-time interaction that a physical whiteboard provides. Additionally, I plan to introduce regular coffee hours and/or lunch meetings within the group to facilitate in-person discussion whenever feasible and in compliance with the university guidelines. Such adaptability is critical for navigating through the evolving educational landscape, and I am committed to the continuous innovation in my teaching and mentoring practices to provide all students with accessible and effective education.